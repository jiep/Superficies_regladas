\documentclass[11pt]{beamer}
\usetheme{Madrid}

\usepackage{fontspec}
\usepackage[spanish,es-nodecimaldot]{babel}
\usepackage{amsmath}
\usepackage{amsfonts}
\usepackage{amssymb}
\usepackage{animate}
\usepackage{graphicx}
\usepackage{hyperref}
\usepackage{color}

\author{José Ignacio Escribano}
\title{Superficies regladas}
\date{23 de abril de 2015}
\institute[URJC]{\includegraphics[width=0.1\linewidth]{logoURJC.jpg}}

\setbeamertemplate{caption}[numbered]

\input{comandos}

\begin{document}
	\begin{frame}[plain]
		\maketitle
	\end{frame}
	
	\begin{frame}{Índice}
		\tableofcontents
	\end{frame}
	
	
	\section{Introducción}
	
	\begin{frame}
		\begin{center}
			\Huge\textbf{\textsf{\textcolor{blue}{Introducción}}}
		\end{center}
	\end{frame}
	
	
	\begin{frame}{Introducción}
		\begin{defi}
			Una superficie es una aplicación
			\begin{equation}
			r : [a,b] \times [c,d] \subseteq \R^2 \to \R^3
			\end{equation}
		\end{defi}
		
		\begin{ejemplo}[de superficie]
			\textcolor{red}{Imagen de superficie}
		\end{ejemplo}	
	\end{frame}
	
	\begin{frame}{Introducción}
		\begin{defi}
			Una superficie reglada es aquella generada por una recta en el espacio llamada generatriz, a lo largo de una curva llamada directriz.
		\end{defi}	
		
		\begin{defi}
			Una superficie es reglada si es generada por una familia uniparamétrica de rectas (llamadas generatrices).\\
			
			La parametrización de una superficie reglada es 
			
			\begin{equation}
			r(u,v) = \rho(u) + v a(u)
			\end{equation}
			
			donde $\rho, a : I = [a,b] \to \R^3$ son dos curvas en el espacio.
		\end{defi}
		
		\begin{defi}
			Un conjunto de curvas de una superficie reglada que interseca a cada generatriz en un punto se llama curva generatriz.
		\end{defi}	
		
	\end{frame}
	
	\begin{frame}{Introducción}
		\begin{defi}
			Una superficie reglada es cilíndrica si es de la forma 
			
			\begin{equation}
			r(u,v) = \rho(u) + v a_0
			\end{equation}
			
			con $a_0 \in \R^3$.
		\end{defi}
		
		\begin{defi}
			Una superficie reglada es cónica si es de la forma 
			
			\begin{equation}
			r(u,v) = \rho_0 + v a(u)
			\end{equation}
			
			con $\rho_0 \in \R^3$.\\
			
			$\rho_0$ es el vértice del cono.
		\end{defi}
	\end{frame}
	
	\begin{frame}{Introducción}
			\begin{defi}
				Una superficie reglada es tangente desarrollable si es de la forma 
				
				\begin{equation}
				r(u,v) = \rho(u) + v \rho'(u)
				\end{equation}
			\end{defi}
			
			\begin{defi}
				Una superficie reglada que cumple que $a(u) \neq 0 \ \ \forall u \in I$ se denomina no cilíndrica. 
			\end{defi}
			
			\begin{defi}
				Una superficie no cilíndrica cuyas generatrices son paralelas a un plano directriz fijo se llama superficie de Catalan.
			\end{defi}
			
	\end{frame}
	
	\begin{frame}{Introducción}
		\begin{teo}[Caracterización de las superficies de Catalan]
			Una superficie reglada $r(u,v) = \rho(u) + v a(u)$ es una superficie de Catalan si y sólo si
			\begin{equation}
			[a(u), a'(a), a''(u)] = 0  \ \ \forall u \in I
			\end{equation}.
		\end{teo}
		
		\begin{defi}
			Una superfie de Catalan se dice conoide si todas las generatrices intersecan una recta constante (el eje del conoide).
		\end{defi}
		
	
	\end{frame}
	
	\section{Algunas superficies regladas}
	
	\begin{frame}
		\begin{center}
			\Huge\textbf{\textsf{\textcolor{blue}{Algunas \\ superficies regladas}}}
		\end{center}
	\end{frame}
	
	\subsection{Superficie cilíndrica}
	
	\begin{frame}{Superficie cilíndrica}
		
	\end{frame}
	
	\subsection{Superficie cónica}
	
	\begin{frame}{Superficie cónica}
		
	\end{frame}
	
	\subsection{Conoides}
	
	\subsubsection{Conoide de Wallis}
	
	\begin{frame}{Conoide de Wallis}
		
	\end{frame}
	
	\subsubsection{Conoide de Plücker}
	
	\begin{frame}{Conoide de Plücker}
		
	\end{frame}
	
	\subsubsection{Conoide de Plücker generalizado}
	
	\begin{frame}{Conoide de Plücker generalizado}
		
	\end{frame}
	
	\subsubsection{Paraguas de Whitney}
	
	\begin{frame}{Paraguas de Whitney}
		
	\end{frame}
	
	\subsection{Otras}
	
	\subsubsection{Banda de Möbius}
	
	\begin{frame}{Banda de Möbius}
		
	\end{frame}
	
	\subsubsection{Helicoide}
	
	\begin{frame}{Helicoide}
		
	\end{frame}
	
	\subsubsection{Hiperboloide}
	
	\begin{frame}{Hiperboloide}
		
	\end{frame}
	
	\section{Aplicaciones}
	
		\begin{frame}
			\begin{center}
				\Huge\textbf{\textsf{\textcolor{blue}{Aplicaciones}}}
			\end{center}
		\end{frame}
	
	\subsection{A la arquitectura}
	
	\subsection{Otras}
	
	
	
	
\end{document}