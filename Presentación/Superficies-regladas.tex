\documentclass[11pt]{beamer}
\usetheme{Madrid}

\usepackage{fontspec}
\usepackage[spanish,es-nodecimaldot]{babel}
\usepackage{amsmath}
\usepackage{amsfonts}
\usepackage{amssymb}
\usepackage{animate}
\usepackage{graphicx}
\usepackage{hyperref}
\usepackage{color}

\author{José Ignacio Escribano}
\title{Superficies regladas}
\date{23 de abril de 2015}
\institute[URJC]{\includegraphics[width=0.1\linewidth]{logoURJC.jpg}}

\setbeamertemplate{caption}[numbered]

\input{comandos}

\begin{document}
	\begin{frame}[plain]
		\maketitle
	\end{frame}
	
	\begin{frame}{Índice}
		\tableofcontents
	\end{frame}
	
	
	\section{Introducción}
	
	\begin{frame}
		\begin{center}
			\Huge\textbf{\textsf{\textcolor{blue}{Introducción}}}
		\end{center}
	\end{frame}
	
	
	\begin{frame}{Introducción}
		\begin{defi}[de superficie]
			\textcolor{red}{Introducir definición}
		\end{defi}
	\end{frame}
	
	\begin{frame}{Introducción}
		\begin{ejemplo}[de superficie]
			\textcolor{red}{Imagen de superficie}
		\end{ejemplo}	
	\end{frame}
	
	\begin{frame}{Introducción}
		\begin{defi}[de superficie reglada]
			\textcolor{red}{Introducir definición}
		\end{defi}	
	\end{frame}
	
	\begin{frame}{Introducción}
		\begin{defi}[de superficie reglada]
			\textcolor{red}{Introducir definiciones de superficies regladas (superficie cilíndrica, superficie cónica, superficies de Catalan, conoides, etc)}
		\end{defi}	
	\end{frame}
	
	\begin{frame}{Introducción}
		\begin{defi}[de superficie doblemente reglada]
			\textcolor{red}{Introducir definición}
		\end{defi}	
	\end{frame}
	
	\section{Algunas superficies regladas}
	
	\begin{frame}
		\begin{center}
			\Huge\textbf{\textsf{\textcolor{blue}{Algunas \\ superficies regladas}}}
		\end{center}
	\end{frame}
	
	\subsection{Superficie cilíndrica}
	
	\begin{frame}{Superficie cilíndrica}
		
	\end{frame}
	
	\subsection{Superficie cónica}
	
	\begin{frame}{Superficie cónica}
		
	\end{frame}
	
	\subsection{Conoides}
	
	\subsubsection{Conoide de Wallis}
	
	\begin{frame}{Conoide de Wallis}
		
	\end{frame}
	
	\subsubsection{Conoide de Plücker}
	
	\begin{frame}{Conoide de Plücker}
		
	\end{frame}
	
	\subsubsection{Conoide de Plücker generalizado}
	
	\begin{frame}{Conoide de Plücker generalizado}
		
	\end{frame}
	
	\subsubsection{Paraguas de Whitney}
	
	\begin{frame}{Paraguas de Whitney}
		
	\end{frame}
	
	\subsection{Otras}
	
	\subsubsection{Banda de Möbius}
	
	\begin{frame}{Banda de Möbius}
		
	\end{frame}
	
	\subsubsection{Helicoide}
	
	\begin{frame}{Helicoide}
		
	\end{frame}
	
	\subsubsection{Hiperboloide}
	
	\begin{frame}{Hiperboloide}
		
	\end{frame}
	
	\section{Aplicaciones}
	
		\begin{frame}
			\begin{center}
				\Huge\textbf{\textsf{\textcolor{blue}{Aplicaciones}}}
			\end{center}
		\end{frame}
	
	\subsection{A la arquitectura}
	
	\subsection{Otras}
	
	
	
	
\end{document}